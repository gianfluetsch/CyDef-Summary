%! Licence = CC BY-NC-SA 4.0

%! Author = gianfluetsch
%! Date = 30. Dez 2021
%! Project = cydef_summary

\section{Cyber Defense}

\subsection{Threat Actors}
Die drei wichtigsten Threat Actors sind \textbf{Script Kiddies, Cyber-Kriminelle} sowie \textbf{APTs}.

\subsubsection{Script Kiddy}

\begin{minipage}{0.5\linewidth}
    \textbf{Motivation}
    \begin{itemize}
        \item Aufmerksamkeit
        \item Anerkennung unter Ihresgleichen
        \item Opportunistisch
    \end{itemize}
\end{minipage}
\begin{minipage}{0.45\linewidth}
    \textbf{Fähigkeiten}\\
    \begin{itemize}
        \item Wenige Fähigkeiten und Kenntnisse
        \item Nutzen verfügbare Hacking Tools
    \end{itemize}
\end{minipage}

\subsubsection{Hacktivisten}

\begin{minipage}{0.5\linewidth}
    \textbf{Motivation}
    \begin{itemize}
        \item Aufmerksamkeit
        \item Anerkennung unter Gleichgesinnten
        \item Protest zum Ausdruck bringen
        \item Gezielt\\
    \end{itemize}
\end{minipage}
\begin{minipage}{0.45\linewidth}
    \textbf{Fähigkeiten}\\
    \begin{itemize}
        \item Variieren sehr stark
        \item Wenig bis sehr Fortgeschritten
    \end{itemize}
\end{minipage}

\textbf{Script Kiddys} und \textbf{Hacktivisten} zielen primär auf direkt exponierte Systeme ab wie beispielsweise Webserver. Sie zeichnen sich dadurch aus, dass sie schnellstmöglich ihren «Smash and Grab-Erfolg» bekannt geben
wollen. Sie dringen in den wenigsten Fällen über längere Zeit in Netzwerke ein.

\subsubsection{Cyber-Kriminelle}

\begin{minipage}{0.5\linewidth}
    \textbf{Motivation}
    \begin{itemize}
        \item Finazieller Art, erpressen sehr viel Geld oder stehlen wertvolle Daten
        \item Opportunistisch
    \end{itemize}
\end{minipage}
\begin{minipage}{0.45\linewidth}
    \textbf{Fähigkeiten}\\
    \begin{itemize}
        \item Hohe Professionalisierung
        \item Insider bieten in Untergrundforen an
        \item Setzen Angriffsmittel gegen eine breite Palette von Zielen ein
    \end{itemize}
\end{minipage}


\subsubsection{Terroristen}
\begin{minipage}{0.5\linewidth}
    \textbf{Motivation}
    \begin{itemize}
        \item Sabotage, Schaden \& Chaos anrichten
        \item Angst verbreiten
        \item Aufmerksamkeit und Rekrutierung
    \end{itemize}
\end{minipage}
\begin{minipage}{0.45\linewidth}
    \textbf{Fähigkeiten}\\
    \begin{itemize}
        \item Angriffe auf kritische Systeme befürchtet, aber noch keine bekannt
        \item Nicht-staatliche, terroristische Gruppen scheinen Fähigkeit für gezielte Cyber-Angriffe noch nicht aufgebaut zu haben\\
    \end{itemize}
\end{minipage}

\textbf{Cyber-Kriminelle} benötigen eine erheblich lange Verweildauer in den Systemen, um an die gewünschten Daten zu kommen und stehen oftmals in ihren technischen Fähigkeiten vielen APTs in nichts nach. Der entscheidende
Unterschied liegt jedoch in den strategischen Zielen dieser Akteure.

\subsubsection{Advanced Persistent Threat (APT)}
\begin{minipage}{0.5\linewidth}
    \textbf{Motivation}
    \begin{itemize}
        \item Klare Mission, höhere, meist staatliche Ziele
        \item Spionage, politisch/ militärisch
        \item Sabotage, Schaden anrichten
        \item Sehr gezielt
    \end{itemize}
\end{minipage}
\begin{minipage}{0.45\linewidth}
    \textbf{Fähigkeiten}\\
    \begin{itemize}
        \item Königsklasse
        \item Ausreichend Ressourcen, staatlich oder staatlich-finanziert
        \item \glqq low-and-slow\grqq, bleiben lange unbemerkt im Netzwerk und dringen erneut, über Alternative Wege ein, wenn Zugriffe blockiert werden
        \item Nutzen eigene Tools, wie auch allgemein verfügbare \glqq off-the-shelf\grqq Hacking Tools und Exploits
        \item Kombinieren mehrere Angriffsmethoden und -techniken
        \item Von Menschen koordinierte Aktionen\\
    \end{itemize}
\end{minipage}

\textbf{APTs} verbleiben in den meisten Fällen \textcolor{red}{\textbf{sehr lange im Netzwerk}} des Opfers (in der Regel mehr als 200 Tage) und betreiben einen entsprechend hohen Aufwand, um ihren Zugang über mehrere Wege sicherzustellen. 
\textbf{Deshalb sollte das gesamte Ausmass des Angriffs unbedingt erst verstanden sein bevor man einen APT Angriff stoppt.}

\subsection{Angriffs Methoden}
Sie verstehen die Denkweise der Angreifer und deren
Methoden und Werkzeuge

\begin{itemize}
    \item Direkte Attacken (Brute-Force)
    \item Indirekte Attacken (Virus, Malware, Ransomware)
    \item Man-in-the-Middle/ Man-in-the-Browser
    \item Privilege Elevation (gain root, gain Administrator)
    \item APT and Data Exfiltration (C\&C Server, Agent, Zombie-Host)
    \item Social Engineering (Identitätsdiebstahl)
    \item MITRE ATT\&CK Framework
\end{itemize}


%! Licence = CC BY-NC-SA 4.0

%! Author = gianfluetsch
%! Date = 30. Dez 2021
%! Project = cydef_summary

\section{Forensic Readiness}

\subsection{Forensic Readiness Apache Reverse Proxy}
Distributed systems, and those using microservices architectures in particular, scatter the order's log messages further, across multiple locations, to be gathered by centralised logging tools. To make it straightforward to investigate problems, you need a way to group the log messages that relate to a particular order, in all of your systems or services.

\subsubsection{Unique-ID}
Correlation IDs provide the standard solution to this problem. A correlation ID uniquely identifies each \glqq customer order\grqq, or the equivalent in your system. Similarly, web-based applications have \glqq user requests\grqq, for each user interaction

\subsubsection{How do you enable the unique-id in the apache?}
By using the \lstinline|mod_unique_id| module.

\subsubsection{How do you log the unique-id into the forensic log}
\lstinline|mod_log_forensic| is included in most distribution packages of Apache and comes with the source tarball download, but if you compile Apache 2.x.x from source, you need to add \lstinline|--enable-log-forensic| and \lstinline|--enable-unique-id| to the configure line.

\subsubsection{How do you inject the unique-id to the backend service?}
By adding the \textit{UNIQUE\_ID} into the following Request Header towards to backend service. On both parts it will be logged for logs/ time correlation.

\subsection{HTTP and HTTPS MitM Apache Reverse Proxy}
This exercise is explaining the so-called online phishing attack. We want to learn how to setup http/https listener, that is forwarding everything to the target webserver. We want to create a \textit{http} and \textit{https} listener and both shall forward to the backend system.
The reverse proxy can be used to forward all traffic to another apache instance within the same docker container.

\subsubsection{Explain the benefit of having a http to https reverse proxy in a phishing campaign}
By using a http to https proxy you ensure having a secured tls connection to the outer world. Having a proxy with unique logging IDs enables for a better traceability for connections especially in a phishing campaign.

\subsubsection{Explain the benefit of having a https to https reverse proxy in a phishing campaign}
By having a https to https proxy as a man-in-the-middle which breaks up tls connections it is able to track connections and its contents for malware/ content filtering and logging with unique IDs. As https reverse proxy internal tls connections are secured by a valid (AD-CA or self-signed distributed) certificate.

\subsubsection{Explain how your reverse proxy online phishing could be advertised to a victim within the same network (LAN)}
With DNS/ ARP spoofing, IPV6 Router Advertisements or Neighbour Discovery fakes.

\subsubsection{Explain how your reverse proxy online phishing could be advertised to a victim over the internet}
With Routing malfunctioning, Domain takeovers, unsecured public DNS. A good mitigation against these attacks is \textit{DNSSec}.

\subsection{Man in the Middle - SSH}
The Man-in-the-Middle (MitM) attack is a cyberattack where the attacker secretly relays and possibly alters the communications between two parties who believe that they are directly communicating with each other. This theory is discussing MitM for the ssh protocol.

\subsubsection{Explain why public/ key auth is really preventing MitM}

\begin{enumerate}
    \item When logging onto the client via ssh on port 4444, it is being forwarded as configured to the ssh server.
    \item The username/pw is being forwarded to the ssh server.
    \item There are no secrets exchanged doing initialization. As the public key is stored on the device, the user is \glqq known\grqq.
    \item The server sends a challenge to the initiating client, the client signs the nonce with the private key and returns it
    to the ssh destination. Identity is proven by this.
\end{enumerate}

\subsubsection{Explain the purpose of editing the ssh client configuration}
The user is identified on the target system by the public key id. The identity is validated by the correct signature of the nonce by the client sent to the server back. Only possible with the correct private key.

\subsubsection{Explain why 2FA would not fix the problem of ssh MitM}
\begin{itemize}
    \item 2FA doesn't help if the purpose are live (on the fly).
    \item If a attacker claims to have the private key and a 2FA message is sent to the (correct) owners smartphone, it does help, as the user wouldn't confirm the request.
\end{itemize}
